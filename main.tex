\documentclass{amsart}

%% Packages

%\usepackage{etoolbox}
%\makeatletter
%\let\ams@starttoc\@starttoc
%\makeatother
%\makeatletter
%\let\@starttoc\ams@starttoc
%\patchcmd{\@starttoc}{\makeatletter}{\makeatletter\parskip\z@}{}{}
%\makeatother

%\usepackage[parfill]{parskip}

\usepackage[colorlinks=true,linkcolor=blue,citecolor=blue,urlcolor=blue]{hyperref}
\usepackage{bookmark}
\usepackage{amsthm,thmtools,amssymb,amsmath,amscd}

\usepackage[bibstyle=alphabetic,citestyle=alphabetic,backend=bibtex]{biblatex}
% Biblatex bug
\makeatletter
\def\blx@maxline{77}
\makeatother
\bibliography{Bibliography}

\usepackage{fancyhdr}
\usepackage{esint}

\usepackage{enumerate}

\usepackage{pictexwd,dcpic}

\usepackage{graphicx}

\usepackage{breqn}

%% Paper specific macros
\DeclareMathOperator{\speed}{h}


%% Theorems

\declaretheorem[name=Theorem,numberwithin=section]{thm}
\declaretheorem[name=Remark,style=remark,sibling=thm]{rem}
\declaretheorem[name=Lemma,sibling=thm]{lemma}
\declaretheorem[name=Proposition,sibling=thm]{prop}
\declaretheorem[name=Definition,style=definition,sibling=thm]{defn}
\declaretheorem[name=Corollary,sibling=thm]{cor}
\declaretheorem[name=Assumption,style=remark,sibling=thm]{ass}
\declaretheorem[name=Example,style=remark,sibling=thm]{example}


\numberwithin{equation}{section}

\usepackage{cleveref}
\crefname{lemma}{Lemma}{Lemmata}
\crefname{prop}{Proposition}{Propositions}
\crefname{thm}{Theorem}{Theorems}
\crefname{cor}{Corollary}{Corollaries}
\crefname{defn}{Definition}{Definitions}
\crefname{example}{Example}{Examples}
\crefname{rem}{Remark}{Remarks}
\crefname{ass}{Assumption}{Assumptions}
\crefname{not}{Notation}{Notation}

%Symbols
\renewcommand{\~}{\tilde}
\renewcommand{\-}{\bar}
\newcommand{\bs}{\backslash}
\newcommand{\cn}{\colon}
\newcommand{\sub}{\subset}

\newcommand{\N}{\mathbb{N}}
\newcommand{\R}{\mathbb{R}}
\newcommand{\Z}{\mathbb{Z}}
\renewcommand{\S}{\mathbb{S}}
\renewcommand{\H}{\mathbb{H}}
\newcommand{\C}{\mathbb{C}}
\newcommand{\K}{\mathbb{K}}
\newcommand{\Di}{\mathbb{D}}
\newcommand{\B}{\mathbb{B}}
\newcommand{\8}{\infty}

%Greek letters
\renewcommand{\a}{\alpha}
\renewcommand{\b}{\beta}
\newcommand{\g}{\gamma}
\renewcommand{\d}{\delta}
\newcommand{\e}{\epsilon}
\renewcommand{\k}{\kappa}
\renewcommand{\l}{\lambda}
\renewcommand{\o}{\omega}
\renewcommand{\t}{\theta}
\newcommand{\s}{\sigma}
\newcommand{\p}{\varphi}
\newcommand{\z}{\zeta}
\newcommand{\vt}{\vartheta}
\renewcommand{\O}{\Omega}
\newcommand{\D}{\Delta}
\newcommand{\G}{\Gamma}
\newcommand{\T}{\Theta}
\renewcommand{\L}{\Lambda}

%Mathcal Letters
\newcommand{\cL}{\mathcal{L}}
\newcommand{\cT}{\mathcal{T}}
\newcommand{\cA}{\mathcal{A}}
\newcommand{\cW}{\mathcal{W}}

%Mathematical operators
\newcommand{\INT}{\int_{\O}}
\newcommand{\DINT}{\int_{\d\O}}
\newcommand{\Int}{\int_{-\infty}^{\infty}}
\newcommand{\del}{\partial}

\newcommand{\inpr}[2]{\left\langle #1,#2 \right\rangle}
\newcommand{\fr}[2]{\frac{#1}{#2}}
\newcommand{\x}{\times}
\newcommand{\abs}[1]{\left|{#1}\right|}
\newcommand{\bdry}[1]{\partial {#1}}
\DeclareMathOperator{\Tr}{Tr}
\newcommand{\tracefree}[1]{\mathring{{#1}}}

\DeclareMathOperator{\intprod}{\iota}
\DeclareMathOperator{\dive}{div}
\DeclareMathOperator{\id}{id}
\DeclareMathOperator{\pr}{pr}
\DeclareMathOperator{\Diff}{Diff}
\DeclareMathOperator{\supp}{supp}
\DeclareMathOperator{\graph}{graph}
\DeclareMathOperator{\osc}{osc}
\DeclareMathOperator{\const}{const}
\DeclareMathOperator{\dist}{dist}
\DeclareMathOperator{\loc}{loc}
\DeclareMathOperator{\grad}{grad}
\DeclareMathOperator{\ric}{Ric}
\DeclareMathOperator{\Rm}{Rm}
\DeclareMathOperator{\weingarten}{\mathcal{W}}
\DeclareMathOperator{\inj}{inj}
\DeclareMathOperator{\Sc}{R}
\DeclareMathOperator{\sff}{A}
\DeclareMathOperator{\einsteinhilbert}{\mathcal{E}}
\DeclareMathOperator{\eulerchar}{\chi}

%Environments
\newcommand{\Theo}[3]{\begin{#1}\label{#2} #3 \end{#1}}
\newcommand{\pf}[1]{\begin{proof} #1 \end{proof}}
\newcommand{\eq}[1]{\begin{equation}\begin{alignedat}{2} #1 \end{alignedat}\end{equation}}
\newcommand{\IntEq}[4]{#1&#2#3	 &\quad &\text{in}~#4,}
\newcommand{\BEq}[4]{#1&#2#3	 &\quad &\text{on}~#4}
\newcommand{\br}[1]{\left(#1\right)}

%Logical symbols
\newcommand{\Ra}{\Rightarrow}
\newcommand{\ra}{\rightarrow}
\newcommand{\hra}{\hookrightarrow}
\newcommand{\mt}{\mapsto}

%Fonts
\newcommand{\mc}{\mathcal}
\renewcommand{\it}{\textit}
\newcommand{\mrm}{\mathrm}

%Spacing
\newcommand{\hp}{\hphantom}


%\parindent 0 pt

\protected\def\ignorethis#1\endignorethis{}
\let\endignorethis\relax
\def\TOCstop{\addtocontents{toc}{\ignorethis}}
\def\TOCstart{\addtocontents{toc}{\endignorethis}}


\title[Isoperimetric Comparison: Ricci Flow with Positive Curvature]{Isoperimetric Comparisons For Ricci Flow on Manifolds with Positive Ricci Curvature}

\curraddr{}
\email{}
\date{\today}

\dedicatory{}
\subjclass[2010]{}
\keywords{}

\begin{document}

\begin{abstract}
\end{abstract}

\maketitle

\section{Introduction}
\label{sec:intro}

Let \(M^n\) be a smooth, \(n\)-dimensional manifold and \(g = g_t\) a smooth, one-parameter family of metrics evolving by the Ricci flow:
\begin{equation}
\label{eq:rf}
\partial_t g = -2 \ric.
\end{equation}

The Isoperimetric Profile is defined by,
\[
I_t (x) = I(x, t) := \inf\{\abs{\bdry{\Omega}}_{g_t} : \abs{\Omega}_{g_t} = x\}
\]
and the infimum ranges over all compactly contained, open sets \(\Omega\) with smooth boundary. An isoperimetric domain is such a set \(\Omega\) satisfying \(I(\abs{\Omega}_{g_t}, t) = \abs{\bdry{\Omega_0}}_{g_t}\).

\section{The Isoperimetric Profile}

In this section we discuss some generalities related to the isoperimetric profile.

\subsection{Small Scale Asymptotics}

\begin{itemize}
\item First order asymptotics are easy. See Ni and Wang ref. The constant is universal, dimension dependent essentially since manifolds are locally Euclidean.
\item Use Alfred Gray's formulae for volume of small geodesic balls to get upper bound.
\item Lower bound is rather more subtle. But one would expect the second order terms just come from constant curvature geometry.
\end{itemize}

\subsection{A Differential Inequality For The Isoperimetric Profile}
\label{sec:iso_diff_ineq}

In this section, we begin by deriving the general formulae for metrics evolving according to
\begin{equation}
\label{eq:dtg}
\partial_t g = 2h
\end{equation}
where \(h \in \Gamma^{\infty}(M, T^{\ast}M \odot T^{\ast} M)\) is a smooth section of symmetric, bilinear forms on \(TM\). We retain the \(2\) simply for the convenience of avoiding factors of \(1/2\) appearing thoughout our formulae below. The case of the Ricci flow is of course \(h = -\ric\).

\subsubsection{Variation Formulae}
\label{subsec:iso_diff_ineq_variation}

Our approach here is variational. Let \(\Omega_0 \subset M\) be an isoperimetric domain at time \(t = t_0\) so that,
\[
I_{t_0} (\abs{\Omega_0}_{t_0}) = \abs{\bdry{\Omega_0}}_{t_0}.
\]
Let also \(\Omega_u\) denote a smooth variation of \(\Omega_0\). That is, there exists an \(\epsilon > 0\) and a smooth map \(\phi: \Omega_0 \times (-\epsilon, \epsilon) \to M\) such that, writing \(\phi_u(\cdot) = \phi(\cdot, u)\) and \(\Omega_u = \phi_u(\Omega_0)\), we have \(\phi_0\) is the inclusion \(\Omega_0 \to M\) and for each \(u\), \(\phi_u\) is an embedding. Let \(\nu\) be the outer unit normal normal vector field to \(\bdry{\Omega_0}\). We consider unit speed, normal variations: \(\phi_{\ast} \partial_u = \nu\) along \(\bdry{\Omega_0}\) which always exist for sufficiently small \(\epsilon\) by working in a tubular neighbourhood of \(\bdry{\Omega_0}\).

\begin{lemma}
\label{lem:spatial_variation}
The first variation of volume and perimeter at \(u = 0\) are
\[
\partial_u \abs{\Omega_u} = \abs{\bdry{\Omega_0}}, \quad \partial_u \abs{\bdry{\Omega_u}} = H_0 \abs{\bdry{\Omega_0}}
\]
where \(H_0\) is the mean curvature of \(\bdry{\Omega_0}\) which is constant since is isoperimetric. The second variations are
\[
\partial_u^2 \abs{\Omega_u} = H_0 \abs{\bdry{\Omega_0}}
\]
and
\[
\partial_u^2 \abs{\bdry{\Omega_u}} = H_0^2 \abs{\bdry{\Omega_0}} - \int_{\bdry{\Omega_0}} \left(\abs{\sff}^2 + \ric(\nu)\right) \sigma.
\]
\end{lemma}

\begin{proof}
See \cite[Chapter 1]{Li:/2012}.
\end{proof}

\begin{lemma}
\label{lem:time_variation}
\[
\partial_t \abs{\Omega_0} = \int_{\Omega_0} \Tr_g \speed \mu,
\]
and
\[
\partial_t \abs{\bdry{\Omega_u}} = \int_{\bdry{\Omega_0}} \Tr_g \speed - \speed(\nu, \nu) \sigma.
\]
\end{lemma}

\begin{proof}
The time variation of \(\abs{\Omega_0}\) may be computed from the standard formula for differentiating a determinant,
\[
\partial_t \mu = \tfrac{1}{2}\Tr(g^{\ast} \otimes \partial_t g) =  \tfrac{1}{2} \Tr (g^{\ast} \otimes 2\speed)\mu = \Tr_g \speed \mu
\]
where \(g^{\ast}\) is the dual metric defined by \(g^{\ast} (\alpha, \beta) = g(\alpha^{\sharp}, \beta^{\sharp})\) and \(\alpha^{\sharp}\) is the metric contraction of \(\alpha\) defined by \(\alpha(X) = g(\alpha^{\sharp}, X)\) for any tangent vector \(X\). The variation of \(\abs{\Omega_0}\) then follow immediately by differentiating under the integral sign.

For the time variation of \(\abs{\bdry{\Omega_0}}\) we use the definition,
\[
\sigma = \iota_{\nu} \mu = \Tr \nu \otimes \mu.
\]
Then,
\begin{equation}
\label{eq:dtsigma}
\partial_t \sigma = \Tr(\partial_t \nu \otimes \mu) + \Tr(\nu \otimes \partial_t \mu).
\end{equation}

The second term is simple,
\begin{equation}
\label{eq:dtsigma2}
\begin{split}
\Tr(\nu \otimes \partial_t \mu) &= \Tr(\nu \otimes \Tr(g^{\ast} \otimes \speed) \mu) \\
&= \Tr(g^{\ast} \otimes \speed) \Tr (\nu \otimes \mu) \\
&= \Tr(g^{\ast} \otimes \speed) \sigma.
\end{split}
\end{equation}

For the first term, we must compute \(\partial_t \nu\). Note that \(\bdry{\Omega_0}\) itself is not changing, whereas the metric is changing, and hence so too is \(\nu\) which is defined via,
\begin{equation}
\label{eq:normalequation}
g_t(\nu, \nu) = 1, \quad g_t(\nu, X) = 0 \text{ for every \(X\) tangent to \(\bdry{\Omega_0}\)}.
\end{equation}

From the first defining equation in \eqref{eq:normalequation} and the evolution of the metric, \eqref{eq:dtg}
\[
0 = \partial_t (g(\nu, \nu)) = (\partial_t g) (\nu, \nu) + 2g(\partial_t \nu, \nu)
\]
so that
\[
g(\partial_t \nu, \nu) = -\speed(\nu, \nu).
\]
Thus we obtain,
\begin{equation}
\label{eq:dtsigma1}
\Tr(\partial_t \nu \otimes \mu) = \iota_{\partial_t\nu} \mu =  g(\partial_t \nu, \nu) \sigma = -\speed(\nu, \nu) \sigma.
\end{equation}

Substitution of \eqref{eq:dtsigma1} and \eqref{eq:dtsigma2} into \eqref{eq:dtsigma} and differentiating under the integral sign again completes the proof.
\end{proof}

\begin{rem}
For the record, the second defining equation in \eqref{eq:normalequation}, the evolution of the metric, \eqref{eq:dtg} and \(\partial_t X = 0\) give
\[
g(\partial_t \nu, X) = -(\partial_t g) (\nu, X) - g(\nu, \partial_t X) = -\speed(\nu, X)
\]
which along with \(g(\partial_t \nu, \nu) = - \speed(\nu, \nu)\) completely determines \(\partial_t \nu\):
\[
\partial_t \nu = -\speed(\nu, \nu) \nu - 2\Tr_g h(\nu, \cdot).
\]
\end{rem}

\subsubsection{Barrier Equation}

\begin{itemize}
\item Change of variables obtains an upper barrier - much stronger in general than viscosity. Can be used to derive the viscosity equation.
\item This perhaps is not needed to be written down though. It's written elsewhere.
\end{itemize}

\subsubsection{Viscosity Equation}
\label{subsec:iso_diff_ineq_viscosity}

\begin{thm}
\label{thm:general_viscosity}
The isoperimetric profile is a viscosity super-solution of,
\[
\begin{split}
\partial_t I &\geq I^2 I'' +  \int_{\bdry{\Omega_0}} \left(\abs{\sff}^2 + \ric(\nu, \nu)\right) \sigma  - I'\int_{\Omega_0} \Tr_g \speed \mu + \int_{\bdry{\Omega_0}} \left(\Tr_g \speed - \speed(\nu, \nu)\right) \sigma \\
&= I^2 I'' +  (I')^2 I + \int_{\bdry{\Omega_0}} \left(Tr_g \speed + R_{\Omega_0} - \speed(\nu, \nu) - \ric(\nu)\right) \sigma \\
&\quad - \left(I'\int_{\Omega_0} \Tr_g \speed \mu + \int_{\bdry{\Omega_0}} R_{\bdry{\Omega_0}}\sigma \right),
\end{split}
\]
where \(\Omega_0 = \Omega_0(x_0, t_0)\) is an isoperimetric domain at \((x_0, t_0)\).
\end{thm}

\begin{rem}
At this level of generality, we cannot say anything more. The first line of each equation depends on the speed \(\speed\), and its relation to the ambient curvature along an (essentially unknown) isoperimetric domain. To go further, we must specify the speed which we do later when we specialise to the Ricci flow. More difficult to deal with is the second line, which also depends on the speed \(\speed\) but cannot be balanced against the other terms unless we can apply some (generally topological) argument to rewrite it as a boundary integral. We also address this issue below in dimensions two and three.
\end{rem}

\begin{proof}
Let \(\phi \leq_{x=x_0, t\leq t_0} I\) be smooth, lower supporting function at \((x_0, t_0)\), \(\Omega_0\) an isoperimetric domain for \((x_0, t_0)\) and \(\Omega_u\) a variation of \(\Omega_0\). Define,
\[
\Phi(u, t) = \abs{\bdry{\Omega_u}}_t - \phi(\abs{\Omega_u}_t, t).
\]
Then \(\Phi\) is smooth and since \(\abs{\bdry{\Omega_u}}_t \geq I(\abs{\Omega_u}_t, t) \geq \phi(\abs{\Omega_u}_t, t)\), we have \(\Phi \geq 0\) for \(u \in (-\epsilon, \epsilon)\) and \(t \leq t_0\) along with \(\Phi(0, t_0) = 0\). Throughout this proof we tacitly make use of the variation formula given in \Cref{lem:spatial_variation} and \Cref{lem:time_variation}.

At \((0, t_0)\), the first variation in \(u\) of \(\Phi\) vanishes giving,
\[
H_0 \abs{\bdry{\Omega_0}} = \partial_u \abs{\bdry{\Omega_u}} = \phi' \partial_u \abs{\Omega_u} = \phi' \abs{\bdry{\Omega_0}},
\]
so that \(\phi' = H_0\). We thus have the identities,
\begin{equation}
\label{eq:viscosity_identities}
\abs{\Omega_0} = x_0, \quad I(x_0, t_0) = \abs{\bdry{\Omega_0}} = \phi(x_0, t_0), \quad H_0 = \phi'(x_0, t_0).
\end{equation}

Also, at \((0, t_0)\), the time variation is non-positive, and the second spatial variation is non-negative, hence
\[
\begin{split}
0 &\geq (\partial_t - \partial_u^2) \Phi \\
&= (\partial_t - \partial_u^2) \abs{\bdry{\Omega}} - \phi'(\partial_t - \partial_u^2) \abs{\Omega_u} + \left(\partial_u \abs{\Omega_u}\right)^2 \phi'' - \partial_t \phi.
\end{split}
\]

From the variation formula, and using equation \eqref{eq:viscosity_identities} we obtain
\begin{align*}
(\partial_t - \partial_u^2) \abs{\bdry{\Omega}} &= \int_{\bdry{\Omega_0}} \left(\Tr_g \speed - \speed(\nu, \nu) + \abs{\sff}^2 + \ric(\nu, \nu)\right) \sigma - (\phi')^2 \phi, \\
(\partial_t - \partial_u^2) \abs{\Omega_u} &= \int_{\Omega_0} \Tr_g \speed \mu - \phi \phi', \\
\intertext{and}
\left(\partial_u \abs{\Omega_u}\right)^2 &= \phi^2.
\end{align*}

Putting this together we arrive at,
\[
\begin{split}
\partial_t \phi &\geq \int_{\bdry{\Omega_0}} \left(\Tr_g \speed - \speed(\nu, \nu) + \abs{\sff}^2 + \ric(\nu, \nu) \right)\sigma - (\phi')^2 \phi \\
&\quad - \phi'\int_{\Omega_0} \Tr_g \speed \mu + \phi (\phi')^2 \\
&\quad + \phi^2 \phi'' \\
&= \phi^2 \phi'' + \int_{\bdry{\Omega_0}} \left(\abs{\sff}^2 + \ric(\nu, \nu)\right) \sigma - \phi'\int_{\Omega_0} \Tr_g \speed \mu + \int_{\bdry{\Omega_0}} \left(\Tr_g \speed - \speed(\nu, \nu)\right) \sigma.
\end{split}
\]
The inequality in the theorem follows by the definition of viscosity super solution.

For the equality in the statement of the theorem, we use the Gauss equation, which states,
\[
2 \ric(\nu) + \abs{\sff}^2 = R_{\Omega_0} + H^2 - R_{\bdry{\Omega_0}}.
\]
Thus we obtain,
\[
\begin{split}
\partial_t \phi - \phi^2 \phi'' &\geq \int_{\bdry{\Omega_0}} \left(\abs{\sff}^2 + \ric(\nu, \nu)\right) \sigma  - \phi'\int_{\Omega_0} \Tr_g \speed \mu + \int_{\bdry{\Omega_0}} \left(\Tr_g \speed - \speed(\nu, \nu)\right) \sigma \\
&= (\phi')^2 \phi - \phi'\int_{\Omega_0} \Tr_g \speed \mu \\
&\quad + \int_{\bdry{\Omega_0}} \left(-\ric(\nu) + R_{\Omega_0} - R_{\bdry{\Omega_0}} + \Tr_g \speed - \speed(\nu, \nu)\right) \sigma
\end{split}
\]
as required.
\end{proof}

We can improve the theorem somewhat in two and three dimensions.

\begin{cor}
\label{cor:low_dim_general_viscosity}
In two dimensions,
\[
\begin{split}
\partial_t I - I^2 I'' - I(I')^2 &\geq \int_{\bdry{\Omega_0}} \left(\Tr_g X + R_{\Omega_0} - \speed(\nu, \nu) - \ric(\nu)\right)\sigma \\
&\quad - I'\int_{\Omega_0} \Tr_g \speed \mu
\end{split}
\]
while in three dimensions,
\[
\begin{split}
\partial_t I - I^2 I'' - I(I')^2 &\geq \int_{\bdry{\Omega_0}} \left(\Tr_g \speed + R_{\Omega_0} - \speed(\nu, \nu) - \ric(\nu)\right)\sigma \\
&\quad - \left(I'\int_{\Omega_0} \Tr_g \speed \mu +  4\pi \chi(\bdry{\Omega_0})\right).
\end{split}
\]
\end{cor}

\begin{proof}
In two dimensions, \(\bdry{\Omega_0}\) is one dimensional, hence \(R_{\bdry{\Omega_0}} = 0\). Substitution in into the second equality in \Cref{thm:general_viscosity} gives the two dimensional result.

In three dimensions, \(\bdry{\Omega_0}\) is two dimensional, hence \(R_{\bdry{\Omega_0}} = 2K\) and the Gauss-Bonnet formula implies,
\[
\int_{\bdry{\Omega_0}} R_{\bdry{\Omega_0}} = 4\pi \chi(\bdry{\Omega_0}).
\]
where \(\chi(\bdry{\Omega_0})\) denotes the Euler characteristic and we used that \(\bdry{\bdry{\Omega_0}} = \emptyset\). The three dimensional result now also follows from substitution in into the second equality in \Cref{thm:general_viscosity}.
\end{proof}

\subsection{Positivity, Concavity And Topology Of Isoperimetric Domains}

\begin{itemize}
\item Positive Ricci curvature gives concavity of the profile
\item Concavity of the profile gives connectedness of isoperimetric domains
\item These follow from the barrier inequality, which is already in \cite{MR875084} so probably is not needed to be proven here. Definitely needs to be stated though!
\item This allows us to conclude that \(\bdry{\Omega}\) is connected (probably need to work manifolds of positive Ricci with universal cover the sphere?) and hence \(\chi(\bdry{\Omega}) \geq 2\pi \chi(\S^2)\). Otherwise, if there are multiple components to the boundary we don't get this bound.
\item The bound is good because this is exactly equality on the round sphere - that is, \(\Omega_0\) is a spherical cap with \(\bdry{\Omega} = \S^2\).
\end{itemize}

\section{The Isoperimetric Profile Under The Ricci Flow}
\label{subsec:flows_ricci}

The Ricci flow is the case where $\speed = -\ric$, and so $\Tr_g \speed = -R$. The viscosity equation in \Cref{thm:general_viscosity} looks quite appealing in this case:
\begin{equation}
\label{eq:viscosity_ricciflow}
\begin{split}
\partial_t I &\geq I^2I'' + I(I')^2 + I'\int_{\Omega_0} R \mu - \int_{\bdry{\Omega_0}} R_{\bdry{\Omega_0}}\sigma \\
&= I^2I'' + I(I')^2 + \einsteinhilbert(\Omega_0) I' - \int_{\bdry{\Omega_0}} R_{\bdry{\Omega_0}}\sigma
\end{split}
\end{equation}
where \(\einsteinhilbert({\Omega_0}) = \int_{\Omega_0} \Sc \mu\) is the Einstein-Hilbert functional.

\subsection*{Two dimensional Ricci Flow}

Using \Cref{cor:low_dim_general_viscosity}, in two dimensions this becomes,
\[
\partial_t I \geq I^2I'' + I(I')^2 + I'\int_{\Omega_0} R \mu.
\]
After an application of the Gauss-Bonnet theorem, the last term becomes,
\[
I'\int_{\Omega_0} R \mu = 4\pi\chi(\Omega_0)I' - 2\int_{\bdry{\Omega_0}} H \sigma I' = 4\pi\chi(\Omega_0)I'- 2 I (I')^2
\]
and so we get,
\begin{equation}
\label{eq:2d_viscosity_ricciflow}
\begin{split}
\partial_t I &\geq I^2I'' - I(I')^2 + 4\pi\chi(\Omega_0)I' \\
&= I^3 (\ln I)'' + 4\pi\chi(\Omega_0)I'.
\end{split}
\end{equation}
Thus we arrive at a rather more usable differential inequality with the only unknown term coming from the topology of isoperimetric domains, which depends on \(x\). Rather complete results for the closed, two-dimensional case are described in \cite{Bryan:/2016,AndrewsBryan:/2010}.

\subsection*{Three dimensional Ricci Flow}

Using \Cref{cor:low_dim_general_viscosity}, in three dimensions we have
\begin{equation}
\label{eq:3d_viscosity_ricciflow}
\partial_t I \geq I^2I'' + I(I')^2 + \einsteinhilbert({\Omega_0})I' - 4\pi\chi(\bdry{\Omega_0}).
\end{equation}


Compared to the two-dimensional case, the second to last term presents some difficulties since it cannot (in general) be written purely in terms of topological invariants. On the other hand, it is the Einstein-Hilbert functional on an isoperimetric domain and thus has nice properties, especially in relation to the Ricci flow that we may be able to exploit.


\section{Comparison Principle}
\label{sec:comparison}

Now we would like to come up with a comparison principle for the isoperimetric profile under the Ricci flow in three dimensions. The first difficulty is that the profile is defined on the varying domain
\[
\{(x, t) : 0 < x < \abs{M_t}, 0 \leq t < T\}
\]
where both \(\abs{M_t}\) and \(T\) depend on the particular one-parameter family of metrics, \(g_t\) under consideration. We overcome this by an appropriate change of \emph{independent variables.} Another difficulty is that the \emph{dependent variable} \(I\) has uncontrolled range, which we overcome by the scaling of the profile. The choice for the scaling is determined by the scaling of the profile itself, while the change of time parameter is chosen to math the spatial change of variables to produce a parabolic rescaling.

\begin{thm}[Comparison Theorem]
\label{thm:comparison}
Let \(\bar{\varphi} : (y, \tau) \in [0, 1] \times [0, \infty) \to \R\) satisfy
\[
\begin{split}
\partial_{\tau} \bar{\varphi} - \bar{\varphi}^2 \bar{\varphi}'' - \bar{\varphi}(\bar{\varphi}')^2 &\leq -\abs{M_t}^{-\tfrac{n-2}{n}} \left[\einsteinhilbert(M_t)\left(\frac{n-1}{n} \bar{\varphi} - y \bar{\varphi}'\right) + \einsteinhilbert(\Omega_0)\bar{\varphi}'\right] \\
&- \abs{M_t}^{-\tfrac{n-3}{n}}\int_{\bdry{\Omega_0}} R_{\bdry{\Omega_0}}\sigma
\end{split}
\]
and define
\[
\varphi(x, t) = \abs{M_t}^{(n-1)/n} \bar{\varphi} \left(\frac{x}{\abs{M_t}}, \int_0^t \abs{M_t}^{-2/n} dt\right).
\]
If \(\varphi(x, 0) \leq I(x, 0)\) for every \(x \in [0, \abs{M_0}]\), then \(\varphi(x, t) \leq I(x, 0)\) for every \(x \in [0, \abs{M_0}]\) and \(t \in [0, T)\).
\end{thm}

\begin{rem}
\label{rem:scaling}
Under scaling the metric,
\[
g \mapsto c^2 g
\]
we have
\begin{align*}
\mu &\mapsto c^n \mu \\
\Sc &\mapsto c^{-2} \Sc \\
\einsteinhilbert &\mapsto c^{n-2} \einsteinhilbert \\
\sigma &\mapsto c^{n-1} \sigma \\
\Sc_{\bdry{\Omega_0}} &\mapsto c^{-2} \Sc_{\bdry{\Omega_0}}.
\end{align*}
In particular
\[
\abs{M_t}^{-(n-2)/n} \einsteinhilbert \mapsto c^{-(n-2)} \abs{M_t}^{(n-2)/n} c^{n-2}\einsteinhilbert
\]
and
\[
\abs{M_t}^{-(n-3)/n} \int_{\bdry{\Omega_0}} R_{\bdry{\Omega_0}}\sigma \mapsto c^{-(n-3)} \abs{M_t}^{-(n-3)/n} \int_{\bdry{\Omega_0}} c^{-2} R_{\bdry{\Omega_0}} c^{n-1}\sigma.
\]
are scale invariant. The left hand side of the differential inequality from the Comparison Theorem \ref{thm:comparison} is manifestly scale invariant, and the scaling just described shows the right hand side is also scale invariant.

The cases \(n = 2\) and \(n = 3\) are particularly special.

For \(n=2\), the boundary term \(\int_{\bdry{\Omega_0}} R_{\bdry{\Omega_0}}\sigma = 0\), \(\abs{M_t}^{-(n-2)/n} = 1\) is already scale invariant, and \(\einsteinhilbert + \int_{\bdry{\empty}} \kappa ds = 4\pi \chi\) is a topological (hence also scale) invariant.

For \(n=3\), the coefficients \(\abs{M_t}^{-(n-2)/n} \einsteinhilbert\) and \(\abs{M_t}^{-(n-3)/n}\) are already scale invariant while \(\int_{\bdry{\Omega_0}} R_{\bdry{\Omega_0}}\sigma = 4\pi \chi(M_t)\) is a topological (hence also scale) invariant.
\end{rem}

\begin{proof}
Let us write \(\lambda = \abs{M_t}^{(n-1)/n}\), \(y = \frac{x}{\abs{M_t}}\) and \(\tau(t) = \int_0^t \abs{M_t}^{-2/n} dt\). Then we have
\[
\begin{split}
\varphi(x, t) &= \lambda \bar{\varphi} (y(x,t), \tau(t)).
\end{split}
\]
Using \(\partial_t \abs{M_t} = \einsteinhilbert(M_t)\), we also have
\begin{align*}
\partial_t \lambda &= \frac{n-1}{n} \abs{M_t}^{-1/n} \einsteinhilbert(M_t) \\
\partial_t y &= -\frac{x}{\abs{M_t}} \abs{M_t}^{-1} \einsteinhilbert(M_t) \\
y' &= \abs{M_t}^{-1} \\
y'' &= 0 \\
\partial_t \tau &= \abs{M_t}^{-2/n}.
\end{align*}

The time derivative is,
\[
\begin{split}
\partial_t \varphi &= (\lambda \partial_t \tau) \partial_{\tau} \bar{\varphi} + (\partial_t \lambda) \bar{\varphi} + (\lambda \partial_t y) \bar{\varphi}' \\
&= \abs{M_t}^{(n-3)/n} \partial_{\tau} \bar{\varphi} + \left(\frac{n-1}{n} \bar{\varphi} - \frac{x}{\abs{M_t}} \bar{\varphi}'\right) \abs{M_t}^{-1/n} \einsteinhilbert(M_t)
\end{split}
\]

The spatial derivatives are
\begin{align*}
\varphi' &= \abs{M_t}^{-1/n} \bar{\varphi}' \\
\varphi'' &= \abs{M_t}^{-(n+1)/n} \bar{\varphi}''
\end{align*}
so that
\[
\begin{split}
\varphi^2 \varphi'' + \varphi(\varphi')^2 &= \abs{M_t}^{2(n-1)/n} \bar{\varphi}^2 \abs{M_t}^{-(n+1)/n} \bar{\varphi}'' + \abs{M_t}^{(n-1)/n} \bar{\varphi} \abs{M_t}^{-2/n} (\bar{\varphi}')^2 \\
&= \abs{M_t}^{(n-3)/n} (\bar{\varphi}^2 \bar{\varphi}'' + \bar{\varphi}(\bar{\varphi}')^2.
\end{split}
\]

Then we have
\begin{equation}
\label{eq:scaled_diff_ineq}
\begin{split}
\partial_t \varphi - \varphi^2 \varphi'' - \varphi(\varphi')^2 &= \abs{M_t}^{(n-3)/n}\left[ \partial_{\tau} \bar{\varphi} -\bar{\varphi}^2 \bar{\varphi}'' - \bar{\varphi}(\bar{\varphi}')^2\right] + \abs{M_t}^{-1/n} \einsteinhilbert(M_t) \left(\frac{n-1}{n} \bar{\varphi} - y\bar{\varphi}'\right) \\
&\leq \einsteinhilbert(\Omega_0) \abs{M_t}^{-1/n} \bar{\varphi}' - \int_{\bdry{\Omega_0}} R_{\bdry{\Omega_0}}\sigma \\
&= \einsteinhilbert(\Omega_0) \varphi' - \int_{\bdry{\Omega_0}} R_{\bdry{\Omega_0}}\sigma.
\end{split}
\end{equation}
with the inequality coming from differential inequality in the hypotheses of the theorem.

The proof now follows by the maximum principle, since equation \eqref{eq:viscosity_ricciflow} gives \(I\) is a viscosity sub-solution whilst equation \eqref{eq:scaled_diff_ineq} gives \(\varphi\) is a viscosity super-solution of the same equation.
\end{proof}
\section{Construction Of Comparison And Geometry Bounds}

The aim is to obtain convergence to a constant sectional curvature manifold, i.e., a space form. To this end, we begin by studying the isoperimetric profile of the round sphere. Then we construct our comparison function by linearising around the sphere.

\subsection{The Isoperimetric Profile Of The Round Sphere}

\begin{itemize}
\item The isoperimetric inequality on the sphere says isoperimetric regions are spherical caps.
\item In dimension \(2\) an explicit expression is easily obtained by inverting the volume of a spherical cap as a function of the angle.
\item In higher dimensions, this seems somewhat tricky to do explicitly. But maybe working implicitly will be enough.
\item The Einstein-Hilbert functional on spherical caps is easily obtained, as is the mean curvature.
\item Then we try to write the Einstein-Hilbert functional as a concave function, \(f_0\) of \(I'\). \textbf{This is key!}
\end{itemize}

In this section, we describe the isoperimetric profile of the round sphere , \(\S^n(r)\) of radius \(r > 0\). This is the expected limit of Riemannian metrics evolving by Ricci flow in positive curvature (or at least on the Riemannian universal cover).

The isoperimetric inequality on the sphere states that for any open set \(\Omega \subset \S^n\) with smooth boundary, we have
\[
\abs{\bdry{\Omega}} \geq \abs{\bdry{B^n}}
\]
where \(B^n \subseteq \S^n\) is a geodesic ball (i.e. a spherical cap) with volume, \(\abs{B^n} = \abs{\Omega}\). Equality occurs if and only if \(\Omega = B^n\). Thus we have
\[
I_{\S^n} (x) = \abs{\bdry{B^n_x}}
\]
where \(\abs{B^n_x} = x\).

To understand the profile more explicitly, it is convenient to describe the round sphere of radius \(r\), \(\S^n(r)\) in polar coordinates:
\[
(\S^n\backslash \{\pm p_0\}, g_{\text{round}(n)}) = (\S^{n-1} \times (0, \pi), r^2 \sin^2 \varphi g_{\text{round}(n-1)} + r^2 d\varphi \otimes d\varphi)
\]
where \(\pm p\) are antipodal points, \(g_{\text{round}(m)}\) denotes the round metric on the \(m\)-dimensional sphere and \(\varphi \in (0, \pi)\).

An open spherical cap of radius \(\sin \varphi\) is then the set
\[
B^n(\varphi) = \S^{n-1} \times [0, \varphi)
\]
with boundary
\[
\bdry{B^n(\varphi)} = \S^{n-1} \times \{\varphi\}.
\]

Then we have
\begin{align*}
V_n(\varphi) &:= \abs{B^n(\varphi)} = A_{n-1} r^n \int_0^{\varphi} \sin^{n-1} (\varphi) d\varphi \\
A_n(\varphi) &:= \abs{\bdry{B^n(\varphi)}} = A_{n-1} r^{n-1} \sin^{n-1} (\varphi)
\end{align*}
where \(A_{n-1} = A_{n}(\pi/2)\) is the \(n-1\)-dimensional volume of the unit, round sphere \(\S^{n-1}(1)\).

Note that
\[
\partial_{\varphi} V_n(\varphi) = A_{n-1} r^n \sin^{n-1} \varphi > 0
\]
for \(\varphi \in (0, \pi)\) and hence the function \(\varphi \mapsto V_n(\varphi)\) is smoothly invertible. Thus we may write
\[
\varphi = V_n^{-1}(x)
\]
and hence the isoperimetric profile of \(\S^n(r)\) is given by
\[
I_{\S^n(r)} (x) = A_{n-1} r^{n-1} \sin^{n-1} (V_n^{-1}(x)).
\]

In two dimensions, we have
\begin{align*}
V_2(\varphi) &:= \abs{B^2(\varphi)} = 2\pi r^2(1-\cos\varphi) \\
A_2(\varphi) &= \abs{\bdry{B^2(\varphi)}} = 2\pi r \sin (\varphi).
\end{align*}
Then
\[
V_2^{-1}(x) = \arccos\left(1 - \frac{x}{2\pi r^2} \right)
\]
and
\[
\begin{split}
I(x) &= 2\pi r \sin(\Phi_2(x)) = 2\pi r \sqrt{1 - \cos^2\left(\arccos\left(1 - \frac{x}{2\pi r^2} \right)\right)} \\
&= \sqrt{4\pi x - \frac{1}{r^2} x^2} \\
&= \sqrt{4\pi x - K_2 x^2}
\end{split}
\]
where \(K_2 = r^{-2}\) is the Gauss curvature of \(\S^2(r)\).

In higher dimensions, it becomes somewhat more difficult to explicitly invert \(V_n\). For example, when \(n = 3\) we have
\[
V_3(\varphi) = 4\pi r^3\left(\frac{1}{2} \varphi - \frac{1}{4}\sin(2 \varphi)\right).
\]
Thus we work implicitly.

The other important quantity we need to understand is the Einstein-Hilbert action over isoperimetric domains. On \(\S^n(r)\), the scalar curvature is constant,
\[
\Sc = \frac{n(n-1)}{r^2}
\]
and hence
\[
\begin{split}
\einsteinhilbert(\varphi) &:= \einsteinhilbert(B^n(\varphi)) = \frac{n(n-1)}{r^2} \abs{B^n(\varphi)} = \frac{n(n-1)}{r^2} V_n(\varphi) \\
&= n(n-1) A_{n-1} r^{n-2} \int_0^{\varphi} \sin^{n-1} \varphi d\varphi.
\end{split}
\]
We also have
\[
\abs{\S^n(r)} = V_n(\pi) = r^n A_{n+1} = r^n \abs{\S^n(1)}
\]
so that
\[
\abs{\S^n(r)}^{-(n-2)/n} \einsteinhilbert(\varphi) = n(n-1)A_{n+1}^{-(n-2)/n} A_n \int_0^{\varphi} \sin^{n-1} \varphi d\varphi.
\]
Let us just remark in passing that this confirms the scale invariance describe in remark \ref{rem:scaling}.

Now the \emph{key idea} is that we may write \(\abs{\S^n(r)}^{-(n-2)/n} \einsteinhilbert(\varphi)\) as a function of \(x\), \(I\) and \(I'\). For this we have
\begin{align*}
x &= V_n(\varphi) = A_{n-1} r^n \int_0^{\varphi} \sin^{n-1} (\varphi) d\varphi \\
I &= A_n(\varphi) = A_{n-1} r^{n-1} \sin^{n-1} \varphi \\
I' &= A_{n-1} r^{n-1} (n-1) \sin^{n-2} (\varphi) \cos(\varphi)\frac{d\varphi}{dx}.
\end{align*}

Note in particular that
\[
I' = (n-1) \cot \varphi = (n-1) H_0
\]
is the mean curvature along \(\bdry{B^n(\varphi)}\) as in the first variation formula.

In two dimensions,
\[
\abs{\S^n(r)}^{-(n-2)/n} \einsteinhilbert(\varphi) = 4\pi (1 - \cos\varphi).
\]
and
\begin{align*}
x &= 2\pi r^2 (1-\cos\varphi) \\
I &= 2\pi r \sin \varphi.
\end{align*}
To compute \(I'\) we use that
\[
\partial_{\varphi} V_n = A_{n-1} r^n \sin^{n-1} \varphi = I
\]
and that \(x = V_n(\varphi)\), hence
\[
I' = \frac{1}{I} \partial_{\varphi} I = \frac{1}{A_{n-1} r^n \sin^{n-1} \varphi} A_{n-1} r^{n-1} (n-1) \sin^{n-2} \varphi \cos \varphi = (n-1) \frac{1}{r} \cot \varphi.
\]

Then
\[
\varphi = \operatorname{arccot}\left(\frac{r}{n-1} I'\right)
\]
and hence
\[
\begin{split}
\abs{\S^n(r)}^{-(n-2)/n} \einsteinhilbert(\varphi) &= 4\pi \left[1 - \cos\left(\operatorname{arccot}\left(\frac{r}{n-1} I'\right)\right)\right] \\
&= 4\pi \left[1 - \frac{r}{n-1} I' \left((\frac{r}{n-1} I')^2 + 1\right)^{-1/2}\right]
\end{split}
\]
\subsection{Comparison Function}

\begin{itemize}
\item The first main thing is we need \(\einsteinhilbert(\Omega_0) \geq f_0(I')\) where \(f_0\) is the sphere function. That is, on the the round sphere we have equality.
\item Concavity of \(f_0\) allows us to discard the error term as a non-negative term when linearising around the sphere profile.
\item We should also compare the \(\einsteinhilbert(M)\) terms with the sphere.
\item Then I think we obtain a linear differential inequality which can be solved more or less explicitly
\item Either that, or we change variables by substituting the sphere profile and seeking a similarity solution.
\end{itemize}

\section{Convergence To Constant Curvature}

\subsection{Curvature Bounds}

\begin{itemize}
\item The asymptotics of the profile should give the second order term is curvature which is controlled (from the comparison theorem) by the second order terms of the comparison function.
\end{itemize}

\subsection{Isoperimetric Constant Bounds}

\begin{itemize}
\item All we need here is that a lower bound of the profile is preserved, which immediately gives a bound on the isoperimetric constant.
\item This constant is also the Sobolev constant which we use in the interpolation inequalities used to obtain convergence.
\end{itemize}


\subsection{Convergence}

\begin{itemize}
\item The proof should now follow as in the two-dimensional case via interpolation which is aided by Isoperimetric constant control.
\end{itemize}

\subsection{}

\section{How It Works For CSF Distance Comparison}

Equation (10) of \cite{AndrewsBryan:01/2011} gives the differential inequality for the comparison function:
\[
\partial_t Z \geq 4 Z'' + \frac{1}{2\pi} \int_{\gamma} \kappa^2 ds (Z - \ell Z') + Z'\int_p^q \kappa^2 ds.
\]
This is without the \(\epsilon\) terms which are just for applying the maximum principle. Here it assumed that \(\gamma\) has been normalised to have fixed length \(L = 2\pi\). This equation is not scale invariant though it's easy enough to obtain the scale invariant version by appropriate placing of \(L/2\pi\).

Compare with the isoperimetric profile under Ricci Flow in the comparison theorem \ref{thm:comparison}
\[
\begin{split}
\partial_{\tau} \bar{\varphi} - \bar{\varphi}^2 \bar{\varphi}'' - \bar{\varphi}(\bar{\varphi}')^2 &\leq -\abs{M_t}^{-\tfrac{n-2}{n}} \left[\einsteinhilbert(M_t)\left(\frac{n-1}{n} \bar{\varphi} - y \bar{\varphi}'\right) + \einsteinhilbert(\Omega_0)\bar{\varphi}'\right] \\
&- \abs{M_t}^{-\tfrac{n-3}{n}}\int_{\bdry{\Omega_0}} R_{\bdry{\Omega_0}}\sigma.
\end{split}
\]

Both equations are essentially of the form
\[
\partial_t u = L u + \bar{Q} (u' - x u) + \Delta Q u' + C
\]
for \(u = u(x, t)\). Here \(L\) is some second order differential operator, \(\bar{Q}\) is the global curvature term and \(\Delta Q\) depends on the optimal domain. Everything is scale invariant.

For the curve shortening flow, to determine a differential inequality for \(\varphi\) we need to deal with the \(\bar{Q}\) and \(\Delta Q\) terms. We have
\[
\bar{Q} = \frac{1}{2\pi}\int_{\gamma} \kappa^2 ds
\]
and
\[
\Delta Q = \int_p^q \kappa^2 ds
\]

For the global term, by H\"older's inequality we have
\[
\bar{Q} = \frac{1}{2\pi} \int_{\gamma} \kappa^2 ds \geq \frac{1}{2\pi} \frac{\left(\int |\kappa| ds\right)^2}{L} \geq \frac{1}{2\pi} \frac{(2\pi)^2}{2\pi} = 1 = \bar{Q_0}
\]
where \(\bar{Q}_0\) is \(\bar{Q}\) on the sphere.

For the local term, again by H\"older's inequality
\[
\Delta Q = \int_p^q \kappa^2 ds \geq \frac{\theta^2}{\ell} = \frac{4 \arccos^2(Z')}{\ell}.
\]

Thus if \(\varphi\) is concave and symmetric, then \(\varphi - \ell \varphi' > 0\) and \(\varphi' > 0\) (for \(\ell \in [0, \pi]\)). Then using our inequalities, if \(\varphi\) satisfies
\[
\partial_t \varphi \leq 4 \varphi'' + (\varphi - \ell \varphi') + \varphi' \frac{4 \arccos^2(\varphi')}{\ell}
\]
then we have our sub-solution.

Now the important observation is that the function \(f_0 = 4 \arccos^2\) is convex and satisfies
\[
\Delta Q_0 = \frac{1}{x} f_0(Z_0')
\]
on the circle with profile \(Z_0\), while
\[
\Delta Q \geq \frac{1}{x} f_0(Z')
\]
in general. We also have
\[
\bar{Q} \geq 1 = \bar{Q}_0.
\]

H\"older's inequality above gave us the inequality case. To see equality for the circle profile, note that for a circle of radius \(1\),
\[
\kappa \equiv 1, \quad L = 2\pi
\]
and so
\[
\bar{Q}_0 = \frac{1}{2\pi} \int_{\gamma} \kappa^2 ds = 1
\]
while
\[
\Delta Q_0 = \int_p^q \kappa^2 ds = \ell.
\]
Note also that on the unit circle, the angle satisfies
\[
\theta = \ell.
\]

Now, the profile at \(\ell \in [0, 2\pi]\) is just the length of the chord joining two points subtending angle \(\theta = \ell\). Thus
\[
Z_0(\ell) = 2 \sin(\ell/2)
\]
Then we have
\[
Z_0'(\ell) = \cos(\ell/2)
\]
so that
\[
\frac{1}{\ell} f_0(Z_0') = \frac{1}{\ell} 4 \arccos^2(\cos(\ell/2)) = \frac{4}{\ell} (\ell/2)^2 = \ell = \Delta Q_0
\]
and we do indeed have equality for the circle profile.

So now we want to find a solution of
\[
\partial_t \varphi \leq 4 \varphi'' + (\varphi - \ell \varphi') + \varphi' \frac{4\arccos^2(\varphi')}{\ell}
\]
and the key to doing so is to expand \(f_0\) to second order in a Taylor series with remainder around the circle profile:
\[
f_0(\varphi') = f_0(Z_0') + f_0'(Z_0')(\varphi' - Z_0') + \frac{1}{2} f_0''(W)(\varphi' - Z_0')^2 \geq f_0(Z_0') + f_0'(Z_0')(\varphi' - Z_0')
\]
where \(W\) lies between \(\varphi'\) and \(Z_0'\) and \(f_0''\geq 0\) by convexity of \(f_0\).

So now we have improved our inequality to
\[
\partial_t \varphi \leq 4 \varphi'' + (\varphi - \ell \varphi') + \frac{1}{\ell} \varphi' \left[f_0(Z_0') + f_0'(Z_0')(\varphi' - Z_0')\right]
\]

Now the first term in our Taylor expansion actually cancels with the \(-\ell\varphi'\) since
\[
f_0(Z_0') = \ell^2
\]
This is not an accident! You have to trace back where the \(\ell\varphi'\) term came from but I don't have time right now. But this cancelling should be true quite generally!

So now we are left with
\[
\partial_t \varphi \leq 4 \varphi'' + \varphi + \frac{1}{\ell} f_0'(Z_0')[(\varphi')^2 - Z_0'\varphi'].
\]

To solve this equation, we make the ansatz
\[
\varphi(\ell, t) = e^t \Phi\left(\frac{Z_0(\ell)}{e^t}\right).
\]
That is, we seek a similarity solution after changing variables \(\ell \mapsto Z_0(\ell)\), \(t \mapsto e^t\). This should not be too surprising. After all, we aim to show that the profile \(Z\) converges exponentially fast to \(Z_0\) and we linearised around \(Z_0\) already.

With these substitutions we get
\begin{align*}
\partial_t \varphi &= e^t \Phi - Z_0 \Phi' \\
\varphi' &= Z_0' \Phi' \\
\varphi'' &= e^{-t} (Z_0')^2 \Phi'' + Z_0'' \Phi'.
\end{align*}
Substitution into our inequality then gives
\[
e^t \Phi - Z_0 \Phi' \leq (4 e^{-t} (Z_0')^2 \Phi'' + 4 Z_0'' \Phi') + (e^t \Phi - \ell Z_0' \Phi') + \frac{1}{\ell} f_0'(Z_0')[(Z_0' \Phi')^2 - Z_0'\Phi'].
\]
Multiplying through by \(e^{-t}\) and rearranging
\[
\begin{split}
0 &\leq -\Phi + Z_0 e^{-t} \Phi' + 4 (e^{-t} Z_0')^2 \Phi'' + 4 Z_0'' e^{-t} \Phi' + \Phi - \ell Z_0' e^{-t}\Phi'+ \frac{e^{-t}}{\ell} f_0'(Z_0')(Z_0' \Phi')^2 - \frac{1}{\ell} f_0'(Z_0') Z_0' e^{-t} \Phi' \\
&= 4 (e^{-t} Z_0')^2 \Phi'' + \frac{e^{-t}}{\ell} f_0'(Z_0') (Z_0' \Phi')^2 \\
&\quad + e^{-t}\Phi'\left[4Z_0'' + (Z_0 - \ell Z_0') - \frac{1}{\ell} f_0(Z_0') Z_0'\right] \\
&= e^{-t} \frac{(Z_0')^2}{Z_0} \left[4e^{-t}Z_0 \Phi'' + \frac{Z_0}{\ell} f_0'(Z_0') (\Phi')^2\right]
\end{split}
\]
since the circle profile satisfies
\[
0 = \partial_t Z_0 = 4e^{-t}Z_0 \Phi'' + \frac{Z_0}{\ell} f_0'(Z_0') (\Phi')^2.
\]

Write
\[
u = e^{-t} Z_0
\]
for the argument to \(\Phi\) and note that since \(Z_0 = 2 \sin(\ell/2)\), \(Z_0'(\ell) = \cos(\ell/2)\) and \(f_0 = 4 \arccos^2\) we obtain
\[
\frac{1}{\ell} Z_0 f_0'(Z_0') = - \frac{1}{\ell} 2 \sin(\ell/2) \frac{8\arccos(\cos(\ell/2))}{\sqrt{1 - \cos^2(\ell/2)}} = - \frac{1}{\ell} 2 \sin(\ell/2) \frac{4\ell}{\sin(\ell/2)} = -8.
\]
Thus we are finally reduced to proving
\[
0 \leq u \Phi'' - 2 (\Phi')^2
\]

I've dropped a term somewhere! It should be
\[
0 \leq u \Phi'' - 2 \Phi'(\Phi' - 1).
\]

Then for any constant \(c\), the solution of
\[
\begin{cases}
u \Phi'' - 2 \Phi'(\Phi' - 1) &= 0 \\
u(0) &= 0
\end{cases}
\]
is exactly the function from \cite{AndrewsBryan:01/2011}:
\[
\Phi(u) = 2 c \arctan(u/2c).
\]
Furthermore the desired asymptotics for
\[
\varphi(\ell, t) = 2c e^t \arctan(\sin(x/2)/2c e^t)
\]
follow immediately from the asymptotics of \(Z_0 = \sin(x/2)\) which has the correct asymptotics because it's the circle profile and from
\[
\arctan(u) = u - \frac{u^3}{3!} + \cdots
\]
so that
\[
\begin{split}
\varphi(\ell, t) &= \sin(\ell/2) - \frac{1}{3!} \frac{1}{4c^2 e^{2t}} \sin(\ell/2)^3 + \cdots \\
&= \ell/2 - \frac{1}{3!} \left(1 - \frac{1}{4c^2 e^{2t}}\right) \ell^3/8 + \cdots.
\end{split}
\]
Not only do we get the correct first order term, but we also get the exponential decay of the next order term to \(1\) - i.e. to the curvature of the unit circle essentially for free! It is this term that bounds \(\kappa^2\) above via the comparison theorem and the rest is just more or less standard convergence.
\end{document}
